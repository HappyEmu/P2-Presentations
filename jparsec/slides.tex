\documentclass{beamer}
\usepackage[utf8x]{inputenc}
\usepackage{amsmath}
\usepackage{amsfonts}
\usepackage{listings}
\usepackage{ucs}
\usetheme{CambridgeUS}
\institute[Universität Bern]{
  \inst{1}
}
\author{Aaron Karper}
\title{JParsec}
\date{\today}
\begin{document}
\frame{\titlepage}
\frame{\frametitle{Contents}\tableofcontents}
\section{What is JParsec?}
\frame{\frametitle{What is Parsing?}
\begin{itemize}
  \item Converting a text into internal representation.
  \item Usually more complex than RegEx.
\end{itemize}
}
\frame{\frametitle{What is JParsec?}
\begin{itemize}
  \item A monadic parser combinator.
  \item Fitted for a top-down approach to parsing.
  \item Declarative, not imperative.
\end{itemize}
}
\section{The Project}
\frame{\frametitle{The Project}
A turing machine description language
\lstinputlisting{example.tm}
}

\section{Building the Parser}
\frame{\frametitle{{\tt Parser<List<Expression>>}}
A TM description is a bunch of expressions separated by {\tt newlines} 
\lstset{language=Java, breaklines=true}
\lstinputlisting{expparser.java}
}
\frame{\frametitle{{\tt .map} }
In order to convert a {\tt Parser<A>} to {\tt Parser<B>}, you have to {\tt 
.map} it with a {\tt Map<A,B>}.
\lstset{language=Java, breaklines=true}
\lstinputlisting{map1.java}
}
\end{document}
